\documentclass{article}
\usepackage[utf8]{inputenc}
\usepackage{polski}
\usepackage{amsfonts}
\usepackage{listings, lstautogobble}
\usepackage{amsmath}
\usepackage{amsthm}
\usepackage{fancyhdr}
\usepackage{graphicx}
\usepackage{hyperref}
\graphicspath{ {./} }

\usepackage[margin=1in]{geometry}
\newgeometry{margin=1in}

\lstset{
	autogobble=true
}

\newtheorem{theorem}{Twierdzenie}
\newcommand{\plot}[1] {
	\includegraphics[width=\textwidth]{#1}
}

\title{\textbf{Pracownia z analizy numerycznej} \\ Sprawozdanie do zadania \textbf{P2.2}}
\author{Prowadzący: dr Rafał Nowak \\ Jakub Zadrożny, Mateusz Hazy}
\date{Wrocław, dnia 12 grudnia 2017}
\begin{document}
	\maketitle
	\section{Wstęp}
	Precyzyjne wyznaczanie położenia na Ziemi jest kluczowym aspektem prawidłowego działania wielu systemów, z których korzystamy na co dzień. GPS (Global Positioning System) jest to system nawigacji satelitarnej pozwalający określać wpółrzędne obiektu. Składa się z trzech segmentów: kosmicznego, naziemnego oraz użytkownika. Niniejsze sprawozdanie ma na celu zbadanie numerycznej oraz praktycznej dokładności wyznaczania położenia użytkownika na podstawie informacji z segmentu kosmicznego. 
	
	\section{Opis problemu}
	
	\par Informacja, jaką użytkownik otrzymuje od poszczególnego satelity składa się z 4 liczb: \newline
		\textbf{$(X, Y, Z)$} - współrzędne satelity w momencie wysłania sygnału \newline
		\textbf{$T_{w}$} - czas wysłania sygnału \newline
		Niech $T_{o}$ będzie czasem odebrania sygnału przez użytkownika.
	Powyższe informacje pozwalają zapisanie równania sfery, charakteryzującego możliwe położenie użytkownika:
	$$ (x - X)^2 + (y - Y)^2 + (z - Z)^2 = \big[c \cdot (T_{o} - T_{w})\big]^2 $$
	\par Do wyznaczenia pozycji teoretycznie wystarczyłyby informacje z 3 satelit, jednak w praktyce zegary satelit i odbiorcy nie są zsynchronizowane. Rozwiązaniem jest wprowadzenie dodatkowej zmiennej $t$ oznaczającej błąd zegara odbiorcy. Prowadzi to do układu równań:
	$$(x-x_1)^2 + (y-y_1)^2 + (z-z_1)^2 - \big[c(t_1-t)\big]^2 =0 $$
	$$(x-x_2)^2 + (y-y_2)^2 + (z-z_2)^2 - \big[c(t_2-t)\big]^2 =0$$
	$$(x-x_3)^2 + (y-y_3)^2 + (z-z_3)^2 - \big[c(t_3-t)\big]^2 =0$$
	$$(x-x_4)^2 + (y-y_4)^2 + (z-z_4)^2 - \big[c(t_4-t)\big]^2 =0$$
	gdzie $t_{i} = T_{oi} - T_{wi}$, c - prędkość światła.
	\par Wszelkie przedstawione w sprawozdaniu metody będą rozwiązywać powyższy układ równań, bądź korzystały z podobnego podejścia.
	
	\section{Rozwiązywanie układu równań}
	
	Niech
$$f_i(x, y, z, t) = (x-x_i)^2 + (y-y_i)^2 + (z-z_i)^2 - \big[c(t_i-t)\big]^2$$
	
	\subsection{Metoda Newtona}
	

	Z uogólnionego twierdzenia Taylora mamy
$$f_i(x + h_x, y + h_y, z + h_z, t + h_t) \approx f_i(x, y, z, t) + h_x \frac{\partial f_i}{\partial x}(x, y, z, t) 
+ h_y \frac{\partial f_i}{\partial y}(x, y, z, t) + h_z \frac{\partial f_i}{\partial z}(x, y, z, t)
+ h_t \frac{\partial f_i}{\partial t}(x, y, z, t)$$

Ponadto

$$ \frac{\partial f_i}{\partial x} = 2(x-x_i) $$
$$ \frac{\partial f_i}{\partial y} = 2(y-y_i) $$
$$ \frac{\partial f_i}{\partial z} = 2(z-z_i) $$
$$ \frac{\partial f_i}{\partial t} = -2c^2(t-t_i) $$

Stąd
$$f_i(x + h_x, y + h_y, z + h_z, t + h_t) \approx f_i(x, y, z, t) + 2h_x(x-x_i) + 2h_y(y-y_i) + 2h_z(z-z_i) -2c^2h_t(t-t_i)$$

Zastosujemy metodę Newtona. Chcemy znaleźć $x, y, z, t$, dla których $f_i(x, y, z, t) = 0$ dla $1 \leq i \leq 4$.
Niech $x_n \in \mathbb{R}^4$ będzie n-tym przybliżeniem metody Newtona. Szukamy $h \in \mathbb{R}^4$, takiego że
dla $x_{n+1} = x_n + h$ zajdzie $f_i(x_{n+1}) = 0$ dla $1 \leq i \leq 4$.

Niech
$$
x_n = 
    \begin{pmatrix}
    x \\
    y \\
    z \\
    t
    \end{pmatrix}
$$

Rozwiążemy układ równań
$$0 = f_1(x, y, z, t) + 2h_x(x-x_1) + 2h_y(y-y_1) + 2h_z(z-z_1) -2c^2h_t(t-t_1) $$
$$0 = f_2(x, y, z, t) + 2h_x(x-x_2) + 2h_y(y-y_2) + 2h_z(z-z_2) -2c^2h_t(t-t_2) $$
$$0 = f_3(x, y, z, t) + 2h_x(x-x_3) + 2h_y(y-y_3) + 2h_z(z-z_3) -2c^2h_t(t-t_3) $$
$$0 = f_4(x, y, z, t) + 2h_x(x-x_4) + 2h_y(y-y_4) + 2h_z(z-z_4) -2c^2h_t(t-t_4) $$

Inaczej
$$
\begin{pmatrix}
x-x_1  & y-y_1 & z-z_1 & -c^2(t-t_1) \\
x-x_2  & y-y_2 & z-z_2 & -c^2(t-t_2) \\
x-x_3  & y-y_3 & z-z_3 & -c^2(t-t_3) \\
x-x_4  & y-y_4 & z-z_4 & -c^2(t-t_4) \\
\end{pmatrix}
\begin{pmatrix}
h_x \\ h_y \\ h_z \\ h_t
\end{pmatrix}
=
-\frac{1}{2}
\begin{pmatrix}
f_1(x, y, z, t) \\
f_2(x, y, z, t) \\
f_3(x, y, z, t) \\
f_4(x, y, z, t)
\end{pmatrix}
$$

Wtedy 
$$
x_{n+1} = x_n + h = x_n + \begin{pmatrix} h_x \\ h_y \\ h_z \\ h_t \end{pmatrix}
$$

	\subsection{Metoda najmniejszych kwadratów}
	W praktyce użytkownik ma dostęp do więcej niż 4 satelit. Pobierając informacje z $n > 4$ satelit otrzymujemy układ równań
	$$ f_{i}(x, y, z, t) = 0 \ dla \ i = 1,2...n $$
	
	Powyższy układ może być sprzeczny, dlatego celem jest znalezienie $ (x, y, z, t) $ takich, że
	
	$$\sum_{1}^{n} f_{i}^2(x, y, z, t) $$ 
	
	będzie minimalna.
	
	Zastosujemy metodę będącą uogólnieniem metody Newtona opisanej powyżej. \newline
	
	Niech
	$$ h_{n} = -(J_{n}^T J_{n})^{-1} J^T F(x_{n})$$

	gdzie \newline
	
	$J_{n}$ - macierz pochodnych cząstkowych w punkcie $x_n$ \newline
	
	$J_{n}[i, j] = \frac{\partial f_i}{\partial x_j} $ \newline
	
	$F(x) \in R^4 \to R^n , F(x) = \big[f_{i}(x)\big]$ \newline

	Wtedy
	$$ x_{n+1} = x_{n} + h_{n} $$
	Jest n-tym przybliżeniem metody.
	
	\subsection{Metoda Algebraiczna}
	
	Zastosujemy inne podejście (nieiteracyjne). Rozważany układ równań wygląda następująco


$$x^2 + y^2 + z^2 -2xx_1 -2yy_1 -2zz_1 + x_1^2 + y_1^2 + z_1^2 = c^2(t^2 -2tt_1 + t_1^2) $$
$$x^2 + y^2 + z^2 -2xx_2 -2yy_2 -2zz_2 + x_2^2 + y_2^2 + z_2^2 = c^2(t^2 -2tt_2 + t_2^2) $$
$$x^2 + y^2 + z^2 -2xx_3 -2yy_3 -2zz_3 + x_3^2 + y_3^2 + z_3^2 = c^2(t^2 -2tt_3 + t_3^2) $$
$$x^2 + y^2 + z^2 -2xx_4 -2yy_4 -2zz_4 + x_4^2 + y_4^2 + z_4^2 = c^2(t^2 -2tt_4 + t_4^2)$$

Po odjęciu czwartego równania stronami od pierwszych trzech otrzymamy
$$-2x(x_1-x_4) -2y(y_1-y_4) -2z(z_1-z_4) + x_1^2 + y_1^2 + z_1^2 - (x_4^2 + y_4^2 + z_4^2) = c^2(-2t(t_1-t_4)+t_1^2-t_4^2)$$ 
$$-2x(x_2-x_4) -2y(y_2-y_4) -2z(z_2-z_4) + x_2^2 + y_2^2 + z_2^2 - (x_4^2 + y_4^2 + z_4^2) = c^2(-2t(t_2-t_4)+t_2^2-t_4^2) $$
$$-2x(x_3-x_4) -2y(y_3-y_4) -2z(z_3-z_4) + x_3^2 + y_3^2 + z_3^2 - (x_4^2 + y_4^2 + z_4^2) = c^2(-2t(t_3-t_4)+t_3^2-t_4^2)$$

\par Otrzymaliśmy więc układ 3 równań liniowych na 4 zmiennych. Układ taki nie posiada jednozanczego rozwiązania, ale - o ile nie jest sprzeczny - posiada rozwiązania parametryczne. W naszym przypadku (dla realnych danych) powinno to być rozwiązanie zależne od jednego parametru. Oznacza to, że pozostałe zmienne można wyrazić jako kombinacje liniowe tego parametru tak, aby dla dowolnej wartości parametru układ równań był spełniony. Ponieważ dla faktycznych danych żadna zmienna nie powinna być z góry ustalona (na trzech równaniach), to możemy założyć, że parametrem jest $t$. Wtedy dla pewnych rzeczywistych $a_x, a_y, a_z, b_x, b_y, b_z$ mamy
$$x = a_xt + b_x $$
$$y = a_yt + b_y $$
$$z = a_zt + b_z$$

Wówczas dla dowolnego $t$ zachodzi
$$-2(a_xt+b_x)(x_1-x_4)-2(a_yt+b_y)(y_1-y_4)-2(a_xt+b_z)(z_1-z_4)+x_1^2+y_1^2+z_1^2-c_0=c^2(-2t(t_1-t_4)+t_1^2-t_4^2) $$
$$-2(a_xt+b_x)(x_2-x_4)-2(a_yt+b_y)(y_2-y_4)-2(a_xt+b_z)(z_2-z_4)+x_2^2+y_2^2+z_2^2-c_0=c^2(-2t(t_2-t_4)+t_2^2-t_4^2) $$
$$-2(a_xt+b_x)(x_3-x_4)-2(a_yt+b_y)(y_3-y_4)-2(a_xt+b_z)(z_3-z_4)+x_3^2+y_3^2+z_3^2-c_0=c^2(-2t(t_3-t_4)+t_3^2-t_4^2)$$

gdzie $c_0 = x_4^2 + y_4^2 + z_4^2$.

\par Otrzymaliśmy zatem równości trzech funkcji liniowych dla każdego argumentu. 

Stąd
$$a_x(x_1-x_4) + a_y(y_1-y_4) + a_z(z_1-z_4) = c^2(t_1-t_4)$$
$$a_x(x_2-x_4) + a_y(y_2-y_4) + a_z(z_2-z_4) = c^2(t_2-t_4)$$
$$a_x(x_3-x_4) + a_y(y_3-y_4) + a_z(z_3-z_4) = c^2(t_3-t_4)$$
$$-2(b_x(x_1-x_4)+b_y(y_1-y_4)+b_z(z_1-z_4)) + x_1^2+y_1^2+z_1^2 - c_0 = c^2(t_1^2 - t_4^2)$$
$$-2(b_x(x_2-x_4)+b_y(y_2-y_4)+b_z(z_2-z_4)) + x_2^2+y_2^2+z_2^2 - c_0 = c^2(t_2^2 - t_4^2)$$
$$-2(b_x(x_3-x_4)+b_y(y_3-y_4)+b_z(z_3-z_4)) + x_3^2+y_3^2+z_3^2 - c_0 = c^2(t_3^2 - t_4^2)$$


Inaczej
$$
\begin{pmatrix}
x_1-x_4 & y_1-y_4 & z_1-z_4 \\
x_2-x_4 & y_2-y_4 & z_2-z_4 \\
x_3-x_4 & y_3-y_4 & z_3-z_4
\end{pmatrix}
\begin{pmatrix}
a_x \\ a_y \\ a_z
\end{pmatrix}
=
c^2
\begin{pmatrix}
t_1-t_4 \\ t_2-t_4 \\ t_3-t_4
\end{pmatrix}
$$
$$
\begin{pmatrix}
x_1-x_4 & y_1-y_4 & z_1-z_4 \\
x_2-x_4 & y_2-y_4 & z_2-z_4 \\
x_3-x_4 & y_3-y_4 & z_3-z_4
\end{pmatrix}
\begin{pmatrix}
b_x \\ b_y \\ b_z
\end{pmatrix}
=
-\frac{1}{2}
\begin{pmatrix}
c^2(t_1^2-t_4^2) - x_1^2+y_1^2+z_1^2 + c_0 \\
c^2(t_2^2-t_4^2) - x_2^2+y_2^2+z_1^2 + c_0 \\
c^2(t_3^2-t_4^2) - x_3^2+y_3^2+z_1^2 + c_0 
\end{pmatrix}
$$

Z ostatnich dwóch równości możemy łatwo wyznaczyć współczynniki $a_x, a_y, a_z, b_x, b_y, b_z$.

Aby otrzymać konkretne rozwiązanie, możemy podstawić otrzymane zależności do czwartego równania. Wtedy otrzymamy równanie kwadratowe jednej zmiennej następującej postaci

$$
  t^2(a_x^2+a_y^2+a_x^2-c^2) + 2t(a_x(b_x-x_4)+a_y(b_y-y_4)+a_z(b_z-z_4)-c^2t_4^2) + (b_x-x_4)^2+(b_y-y_4)^2+(b_z-z_4)-c^2t_4^2 = 0
$$

Po rozwiązaniu równania otrzymamy dwie możliwości na $t$, które wyznaczą dwa możliwe rozwiązania $x, y, z, t$. W poniższym kodzie przyjęto, że szukanym rozwiązaniem jest to o mniejszym bezwzględnym błędzie zegara.

\subsection{Heurystyka}
Mamy układ równań
$$ f_{i}(x, y, z, t) =0 \ dla \ i = 1,2,3,4$$
Sprowadzamy do układu równań liniowych odejmując ostatnie równanie od pozostałych
$$-2x(x_i-x_4) -2y(y_i-y_4) -2z(z_i-z_4) + x_i^2 + y_i^2 + z_i^2 - (x_4^2 + y_4^2 + z_4^2) = c^2(-2t(t_i-t_4)+t_i^2-t_4^2)$$

Inaczej
$$
\begin{pmatrix}
x_1-x_4 & y_1-y_4 & z_1-z_4 & t_4-t_1 \\
x_2-x_4 & y_2-y_4 & z_2-z_4 & t_4-t_2 \\
x_3-x_4 & y_3-y_4 & z_3-z_4 & t_4-t_3
\end{pmatrix}
\begin{pmatrix}
x \\ y \\ z \\ t
\end{pmatrix}
=
-\frac{1}{2}
\begin{pmatrix}
-(x_1^2+y_1^2+z_1^2-t_1^2)+c_1 \\
-(x_2^2+y_2^2+z_2^2-t_2^2)+c_1 \\
-(x_3^2+y_3^2+z_3^2-t_3^2)+c_1 \\
\end{pmatrix}
$$
gdzie $c_1=x_4^2+y_4^2+z_4^2-t_4^2$.

\section{Testy}

Na poszczególne metody mają wpływ czynniki takie jak:
	\begin{itemize}
		\item wielkość błąd zegara
		\item liczba iteracji
		\item liczba satelit, z których została pobrana informacja
	\end{itemize}
W praktyce na dokładność pomiaru istotny wpływ mają czynniki zewnętrzne:
	\begin{itemize}
		\item opoźnienie światła w troposferze
		\item interferencja fal elektromagnetycznych
	\end{itemize}
Z tego powodu sprawdzanie dokładności metod zostało podzielone na badanie wpływu poszczególnych czynników na precyzję oraz uproszczoną symulację określania położenia w warunkach naturalnych.

\subsection{Symulacja ścieżki}

\par Do sprawdzenia dokładności metod została stworzona symulacja poruszania się użytkownika po losowej ścieżce na małym obszarze Ziemi. Model został uproszczony. Satelity były nieruchome, a błąd powodowany czynnikami zewnętrznymi symulowany był losowym opóźnieniem prędkości światła.


\subsection{Błąd zegara}



\subsection{Liczba iteracji}


\subsection{Liczba satelit}

Zaletą metody najmniejszych kwadratów jest rosnąca odporność na błedy wynikające z czynników zewnętrznych. Potencjalny duży błąd informacji z jednej z satelit jest niwelowany dokładnymi informacjami z pozostałych.

\section{Wnioski}

	 	 
\end{document}