%----------------------------------------------------------------------------------------
%	PACKAGES AND DOCUMENT CONFIGURATIONS
%----------------------------------------------------------------------------------------

\documentclass{article}

\usepackage{graphicx} % Required for the inclusion of images

\usepackage[utf8]{inputenc}
\usepackage[OT4,plmath]{polski}

\usepackage{caption}
\usepackage{subcaption}
\usepackage{amsmath,amssymb,amsfonts,amsthm,mathtools}
\usepackage{xfrac}

\usepackage{hyperref}
\usepackage{url}
\usepackage{array}
\usepackage[flushleft]{threeparttable}

\usepackage{comment}

\usepackage{listings, lstautogobble}
\usepackage{fancyhdr}
\usepackage{graphicx}
\usepackage{hyperref}

\newtheorem{theorem}{Twierdzenie}
\newtheorem{lemma}{Lemat}
\newcommand{\plot}[1] {
	\includegraphics[width=\textwidth]{#1}
}

\graphicspath{ {./} }
\setlength\parskip{4pt}
\setlength\parindent{0pt}
% \setlength{\tabcolsep}{8pt}
% \renewcommand{\arraystretch}{1.5}

%\usepackage{times} % Uncomment to use the Times New Roman font

%----------------------------------------------------------------------------------------
%	DOCUMENT INFORMATION
%----------------------------------------------------------------------------------------

\title{\LARGE\textbf{Pracownia z analizy numerycznej} \\ Sprawozdanie do zadania \textbf{P2.2} \\
\vskip 0.2cm \large Prowadzący: dr Rafał Nowak\\
\author{ Jakub \textsc{Zadrożny}, Mateusz \textsc{Hazy}}}
\date{Wrocław, \today}

\begin{document}
\maketitle

\section{Wstęp}
Precyzyjne wyznaczanie położenia na Ziemi jest problemem, z którym ludzkość zmaga się od setek lat.
Na początku wykorzystywano zupełnie prymitywne metody (jak obserwacja otoczenia), z czasem wypracowano bardziej
zaawansowane sposoby (np. oparte na obserwacjach astronomicznych), aż w końcu opracowano rozwiązanie niemal doskonałe.
Rozwiązaniem tym jest system GPS (\textit{Global Positioning System}), czyli program nawigacji satelitarnej,
pozwalający określać położenie obiektu z dokładnością do kilku metrów (czasem nawet dokładniej).
Składa się on z trzech segmentów: kosmicznego, naziemnego oraz użytkownika. Niniejsze sprawozdanie skupia się na problemach
segmentu użużytkownika i ma na celu przedstawienie oraz porównanie metod wyznaczania położenia obiektu na podstawie
informacji z segmentu kosmicznego.

\subsection{Opis problemu}
Na potrzeby doświadczeń przyjmiemy model, w którym informacja, jaką użytkownik otrzymuje od satelity,
składa się z 4 liczb:
\begin{itemize}
    \item $(X, Y, Z)$ - współrzędne satelity w momencie wysłania sygnału
	\item $T_w$ - czas wysłania sygnału
\end{itemize}
Niech $T_{o}$ będzie czasem odebrania sygnału przez użytkownika. Wtedy czas podróży sygnały wynosi $T_o - T_w$ i
przyjmując, że sygnał przemieszcza się stale z prędkością światła, możemy określić, w jakiej odległości od satelity
znajduje się odbiornik (w linii prostej). Innymi słowy, odbiornik leży na \textit{sferze} o środku w punkcie $(X, Y, Z)$
i promieniu $c(T_w-T_o)$. Jeżeli pozycję odbiornika oznaczymy przez $(x, y, z)$, to równanie tej sfery wygląda następująco
\[
(x - X)^2 + (y - Y)^2 + (z - Z)^2 = \big[c(T_{o} - T_{w})\big]^2
\]

Posiadając informacje z trzech satelit, otrzymalibyśmy teoretycznie równania trzech sfer, które przecinałyby się w
mniej więcej dwóch punktach, z których jeden leżałby w pobliżu Ziemi. Jednak w praktyce zegary satelit i odbiornika
nie są zsynchronizowane, przez co promień każdej sfery zostanie zostanie przekłamany i pozycja zostanie wyznaczona błędnie
(lub sfery w ogóle nie będą mieć punktów wspólnych). Rozwiązaniem tego problemu jest wprowadzenie dodatkowej zmiennej $t$
oznaczającej błąd zegara odbiorcy. Aby otrzymać jednoznaczne rozwiązanie potrzebujemy teraz dodatkowego równania.
Prowadzi to do układu
\begin{equation} \label{eq:main_system}
\begin{aligned}
    (x-x_1)^2 + (y-y_1)^2 + (z-z_1)^2 - \big[c(t_1-t)\big]^2 &= 0 \\
    (x-x_2)^2 + (y-y_2)^2 + (z-z_2)^2 - \big[c(t_2-t)\big]^2 &= 0 \\
    (x-x_3)^2 + (y-y_3)^2 + (z-z_3)^2 - \big[c(t_3-t)\big]^2 &= 0 \\
    (x-x_4)^2 + (y-y_4)^2 + (z-z_4)^2 - \big[c(t_4-t)\big]^2 &= 0
\end{aligned}
\end{equation}
gdzie $(x_i, y_i, z_i)$ to współrzędnego i-tego satelity w czasie wysłania sygnału, $t_i$ to czas podróży sygnału od
i-tego satelity oraz $c$ -- prędkość światła.
Celem niniejszego sprawodzdania jest rozważenie możliwości rozwiązania tego układu.

\section{Metody rozwiązywania układu}
Źródła donoszą, że w początkowym okresie działania systemu \textit{GPS} do rozwiązywania układu (\ref{eq:main_system})
używano metod iteracyjnych (metody Newtona). Metody nieiteracyjne pojawiły się nieco później, a pierwszą z nich
zaproponował w swojej pracy Stephen Bancroft w roku 1985. Od tego czasu opublikowano na ten temat wiele prac, w których
stale rozwija się istniejące metody oraz porównuje się ich przydatnośc w rozmaitych sytuacjach praktycznych.

W niniejszym sprawozdaniu wyprowadzono szczegółowo dwa podstawowe podejścia: iteracyjne (metodę Newtona) oraz
algebraiczne (nieco odmienne niż zaproponowane przez S. Bancrofta, jednak o podobnym charakterze).

\subsection{Metoda Newtona}
Niech $d_i \coloneqq ct_i$, a pod zmienną $t$ podstawmy $ct$.
Wprowadźmy oznaczenia
\begin{align}
f_i(x, y, z, t) &\coloneqq (x-X_i)^2 + (y-Y_i)^2 + (z-Z_i)^2 - (D_i-t)^2 \\
\label{eq:F} F(x, y, z, t) &\coloneqq (f_1(x, y, z, t) \ f_2(x, y, z, t) \ f_3(x, y, z, t) \ f_4(x, y, z, t))^T
\end{align}
dla $1 \leq i \leq 4$, zgodnie z notacją układu (\ref{eq:main_system}), gdzie $T$ oznacza transpozycję.

Opracujemy metodę rozwiązania układu na podstawie metody Newtona. Chcemy znaleźć miejsce zerowe funkcji
$F$ określonej przez (\ref{eq:F}). Załóżmy, że mamy n-te przybliżenie wyniku $x_n \in \mathbb{R}^4$.
Checmy znaleźć takie $h \in \mathbb{R}^4$, że dla $x_{n+1} = x_n + h$ zajdzie $F(x_{n+1}) = 0$.

\begin{lemma}(Linearyzacja) \label{th:linearize}
    Dla funkcji $f: \mathbb{R}^n \rightarrow \mathbb{R}$ różniczkowalnej na przedziale $[a, b]$
    oraz punktów $c, x \in \mathbb{R}^n$ z przedziału $[a, b]$ zachodzi
    \[
        f(x) \approx f(c) + \nabla f(c) \circ (x - c)
    \]
\end{lemma}
\begin{proof}
    Dowód dostępny jest w literaturze, m.in. w pozycji \ref{lit:linearization}.
\end{proof}

Korzystając z lematu \ref{th:linearize} otrzymujemy
\[
f_i(x+h) \approx f_i(x) + \nabla f_i(x) \circ h
\]
gdzie $x, h \in \mathbb{R}^4$.
Linearyzując funkcje $f_i$ po kolei otrzymamy więc
\[
F(x+h) \approx F(x) + \mathbf{J}(x)h
\]
gdzie $\mathbf{J}$ to jakobian funkcji $F$.

Rożniczkując funkcje $f_i$ po kolejnych zmiennych otrzymujemy, że dla $p \in \mathbb{R}^4,\ p=(x \ y \ z \ t)^T$
\[
\mathbf{J}(p) = 2
\begin{pmatrix}
    x-X_1  & y-Y_1 & z-Z_1 & T_1-t \\
    x-X_2  & y-Y_2 & z-Z_2 & T_2-t \\
    x-X_3  & y-Y_3 & z-Z_3 & T_3-t \\
    x-X_4  & y-Y_4 & z-Z_4 & T_4-t
\end{pmatrix}
\]

Aby sprawdzić, kiedy $F(x_{n+1}) = 0$, rozwiążemy układ równań liniowych
\begin{align*}
    0 &= F(x_n) + \mathbf{J}(x_n)h \\
    -F(x_n) &= \mathbf{J}(x_n)h \\
    h &= -\mathbf{J}(x_n)^{-1}F(x_n)
\end{align*}
Możemy określić $x_{n+1} \coloneqq x_n + h$.

Ponieważ ustalamy pozycję głównie obiektów położonych na Ziemi, więc usatlimy przybliżenie początkowe jako
wektor $x_0 = (0 \ 0 \ 0 \ 0)^T$.

\textbf{Uwaga.} \enspace W implementacji niezbędne jest odwrócenie jakobianu funkcji $F$ wyliczonego w
pewnym przybliżeniu. W niniejszym rozwiązaniu osiągnięto to za pomocą funkcji bibliotecznej \texttt{\textbackslash}
języka \texttt{Julia}.

\subsection{Metoda algebraiczna}
Wyprowadzimy metodę opierającą się wyłącznie na podstawowych prawach algebry liniowej.
Układ równań (\ref{eq:main_system}) po wykonaniu mnożeń wygląda następująco
\begin{equation}
\begin{aligned}
    x^2 + y^2 + z^2 -2xX_1 -2yY_1 -2zZ_1 + c_1 = t^2 -2tD_1 \\
    x^2 + y^2 + z^2 -2xX_2 -2yY_2 -2zZ_2 + c_2 = t^2 -2tD_2 \\
    x^2 + y^2 + z^2 -2xX_3 -2yY_3 -2zZ_3 + c_3 = t^2 -2tD_3 \\
    x^2 + y^2 + z^2 -2xX_4 -2yY_4 -2zZ_4 + c_4 = t^2 -2tD_4
\end{aligned}
\end{equation}
gdzie $c_i=X_i^2+Y_i^2+Z_i^2-D_i^2$ dla $1 \leq i \leq 4$.

Po odjęciu czwartego równania stronami od pierwszych trzech otrzymamy
\begin{equation}
\begin{aligned}
    \label{eq:modified_system}
    -2\big[ x\Delta_{X,1} + y\Delta_{Y, 1} + z\Delta_{Z, 1} ] + \Delta_{c, 1} = -2t\Delta_{D, 1} \\
    -2\big[ x\Delta_{X,2} + y\Delta_{Y, 2} + z\Delta_{Z, 2} ] + \Delta_{c, 2} = -2t\Delta_{D, 2} \\
    -2\big[ x\Delta_{X,3} + y\Delta_{Y, 3} + z\Delta_{Z, 3} ] + \Delta_{c, 3} = -2t\Delta_{D, 3}
\end{aligned}
\end{equation}
gdzie $\Delta_{A, i} = A_i - A_4$ dla $A \in \{X, Y, Z, D, c\}$ oraz $1 \leq i \leq 4$.

Jest to układ 3 równań liniowych na 4 zmiennych. Układ taki nie posiada jednozanczego rozwiązania,
ale -- o ile nie jest sprzeczny -- posiada rozwiązania parametryczne. W przypadku układu (\ref{eq:modified_system}),
gdy dane z satelit nie są liniowo zależne, będą istnieć rozwiązania zależne od dokładnie jednego parametru.
Oznacza to, że pozostałe zmienne można wyrazić jako kombinacje liniowe tego parametru (i stałej) tak,
aby dla dowolnej wartości parametru układ równań był spełniony. W układzie (\ref{eq:modified_system}) dla rzeczywistych
danych żadna zmienna nie powinna być z góry stalona, zatem możemy założyć, że parametrem jest $t$.
Wtedy dla pewnych rzeczywistych $a_x, a_y, a_z, b_x, b_y, b_z$ mamy
\begin{equation}
\begin{aligned}
    \label{eq:x_from_t}
    x = a_xt + b_x \\
    y = a_yt + b_y \\
    z = a_zt + b_z
\end{aligned}
\end{equation}
oraz dla dowolnego $t$ zachodzi
\begin{equation}
\begin{aligned}
    \label{eq:t_functions}
    (a_xt+b_x)\Delta_{X,1} + (a_yt+b_y)\Delta_{Y,1} + (a_zt+b_z)\Delta_{Z, 1} -\sfrac{1}{2}\Delta_{c, 1} =t\Delta_{D,1} \\
    (a_xt+b_x)\Delta_{X,2} + (a_yt+b_y)\Delta_{Y,2} + (a_zt+b_z)\Delta_{Z, 2} -\sfrac{1}{2}\Delta_{c, 2} =t\Delta_{D,2} \\
    (a_xt+b_x)\Delta_{X,3} + (a_yt+b_y)\Delta_{Y,3} + (a_zt+b_z)\Delta_{Z, 3} -\sfrac{1}{2}\Delta_{c, 3} =t\Delta_{D,3}
\end{aligned}
\end{equation}

Układ (\ref{eq:t_functions}) oznacza równość trzech funkcji liniowych na całej prostej, co
implikuje równość jej współczynników. Stąd
\begin{equation}
\begin{aligned}
    a_x\Delta_{X, 1} + a_y\Delta_{Y, 1} + a_z\Delta_{Z, 1} &= \Delta_{D, 1} \\
    a_x\Delta_{X, 2} + a_y\Delta_{Y, 2} + a_z\Delta_{Z, 2} &= \Delta_{D, 2} \\
    a_x\Delta_{X, 3} + a_y\Delta_{Y, 3} + a_z\Delta_{Z, 3} &= \Delta_{D, 3} \\
    b_x\Delta_{X, 1} + b_y\Delta_{Y, 1} + b_z\Delta_{Z, 1} &= \sfrac{1}{2}\Delta_{c, 1} \\
    b_x\Delta_{X, 2} + b_y\Delta_{Y, 2} + b_z\Delta_{Z, 2} &= \sfrac{1}{2}\Delta_{c, 2} \\
    b_x\Delta_{X, 3} + b_y\Delta_{Y, 3} + b_z\Delta_{Z, 3} &= \sfrac{1}{2}\Delta_{c, 3}
\end{aligned}
\end{equation}

Niech
\[
A \coloneqq
\begin{pmatrix}
\Delta_{X,1} & \Delta_{Y,1} & \Delta_{Z, 1} \\
\Delta_{X,2} & \Delta_{Y,2} & \Delta_{Z, 2} \\
\Delta_{X,3} & \Delta_{Y,3} & \Delta_{Z, 3}
\end{pmatrix}
\]

Wtedy
\begin{equation}
    \label{eq:matrices}
    A
    \begin{pmatrix}
        a_x \\ a_y \\ a_z
    \end{pmatrix}
    =
    \begin{pmatrix}
        \Delta_{D, 1} \\ \Delta_{D, 2} \\ \Delta_{D, 3}
    \end{pmatrix}
    \quad \textrm{oraz} \quad
    A
    \begin{pmatrix}
        b_x \\ b_y \\ b_z
    \end{pmatrix}
    =
    -\frac{1}{2}
    \begin{pmatrix}
        \Delta_{c, 1} \\
        \Delta_{c, 2} \\
        \Delta_{c, 3}
    \end{pmatrix}
\end{equation}

Z równości (\ref{eq:matrices}) możemy wyznaczyć współczynniki $a_x, a_y, a_z, b_x, b_y, b_z$.
Następnie, aby otrzymać konkretne rozwiązanie, podstawimy otrzymane zależności do czwartego równania układu
(\ref{eq:modified_system}) otrzymując równanie kwadratowe jednej zmiennej następującej postaci
\begin{equation} \label{eq:quadratic}
\begin{split}
  t^2(a_x^2&+a_y^2+a_x^2-1) + \\
  &+ 2t\big[ a_x(b_x-X_4)+a_y(b_y-Y_4)+a_z(b_z-Z_4)+D_4^2\big] + \\
  &+ (b_x-X_4)^2+(b_y-Y_4)^2+(b_z-Z_4)^2-D_4^2 = 0
\end{split}
\end{equation}

Po rozwiązaniu równania (\ref{eq:quadratic}) otrzymamy dwa kandydaty na $t$, które
na podstawie zależności (\ref{eq:x_from_t}) wyznaczą dwa rozwiązania $x, y, z, t$
W rozwiązaniu przyjęto, że szukanym wektorem jest ten o mniejszym bezwzględnym błędzie zegara ($t$).

\textbf{Uwaga.} \enspace W implementacji niezbędne jest rozwiązanie pewnych układów równań liniowych
oraz znalezienie miejsc zerowych funkcji kwadratowej. W niniejszym rozwiązaniu do układów równań zastosowano
funkcję biblioteczną \texttt{\textbackslash} języka \texttt{Julia}, natomiast do równań kwadratowych funkcję \texttt{roots}
z pakietu \texttt{Polynomials}.

\section{Podstawowe testy}

\section{Uogólnione metody}
W praktyce użytkownik ma dostęp do więcej niż 4 satelit. Pobierając informacje z $n > 4$ satelit otrzymujemy układ równań
$$ f_{i}(x, y, z, t) = 0 \ dla \ i = 1,2...n $$

Powyższy układ może być sprzeczny, dlatego celem jest znalezienie $ (x, y, z, t) $ takich, że

$$\sum_{1}^{n} f_{i}^2(x, y, z, t) $$

będzie minimalna.

\subsection{Metoda najmniejszych kwadratów}
Zastosujemy metodę będącą uogólnieniem metody Newtona opisanej powyżej. \newline

Niech
$$ h_{n} = -(J_{n}^T J_{n})^{-1} J^T F(x_{n})$$

gdzie \newline

$J_{n}$ - macierz pochodnych cząstkowych w punkcie $x_n$ \newline

$J_{n}[i, j] = \frac{\partial f_i}{\partial x_j} $ \newline

$F(x) \in R^4 \to R^n , F(x) = \big[f_{i}(x)\big]$ \newline

Wtedy
$$ x_{n+1} = x_{n} + h_{n} $$
Jest n-tym przybliżeniem metody.

\subsection{Heurystyka}
Mamy układ równań
$$ f_{i}(x, y, z, t) =0 \ dla \ i = 1,2,3,4$$
Sprowadzamy do układu równań liniowych odejmując ostatnie równanie od pozostałych
$$-2x(x_i-x_4) -2y(y_i-y_4) -2z(z_i-z_4) + x_i^2 + y_i^2 + z_i^2 - (x_4^2 + y_4^2 + z_4^2) = c^2(-2t(t_i-t_4)+t_i^2-t_4^2)$$

Inaczej
$$
\begin{pmatrix}
x_1-x_4 & y_1-y_4 & z_1-z_4 & t_4-t_1 \\
x_2-x_4 & y_2-y_4 & z_2-z_4 & t_4-t_2 \\
x_3-x_4 & y_3-y_4 & z_3-z_4 & t_4-t_3
\end{pmatrix}
\begin{pmatrix}
x \\ y \\ z \\ t
\end{pmatrix}
=
-\frac{1}{2}
\begin{pmatrix}
-(x_1^2+y_1^2+z_1^2-t_1^2)+c_1 \\
-(x_2^2+y_2^2+z_2^2-t_2^2)+c_1 \\
-(x_3^2+y_3^2+z_3^2-t_3^2)+c_1 \\
\end{pmatrix}
$$
gdzie $c_1=x_4^2+y_4^2+z_4^2-t_4^2$.

\section{Rozszerzone testy}

Na poszczególne metody mają wpływ czynniki takie jak:
	\begin{itemize}
		\item wielkość błąd zegara
		\item liczba iteracji
		\item liczba satelit, z których została pobrana informacja
	\end{itemize}
W praktyce na dokładność pomiaru istotny wpływ mają czynniki zewnętrzne:
	\begin{itemize}
		\item opoźnienie światła w troposferze
		\item interferencja fal elektromagnetycznych
	\end{itemize}
Z tego powodu sprawdzanie dokładności metod zostało podzielone na badanie wpływu poszczególnych czynników na precyzję oraz uproszczoną symulację określania położenia w warunkach naturalnych.

\subsection{Symulacja ścieżki}

\par Do sprawdzenia dokładności metod została stworzona symulacja poruszania się użytkownika po losowej ścieżce na małym obszarze Ziemi. Model został uproszczony. Satelity były nieruchome, a błąd powodowany czynnikami zewnętrznymi symulowany był losowym opóźnieniem prędkości światła.


\subsection{Błąd zegara}



\subsection{Liczba iteracji}


\subsection{Liczba satelit}

Zaletą metody najmniejszych kwadratów jest rosnąca odporność na błedy wynikające z czynników zewnętrznych. Potencjalny duży błąd informacji z jednej z satelit jest niwelowany dokładnymi informacjami z pozostałych.

\section{Wnioski}


\end{document}
