%----------------------------------------------------------------------------------------
%	PACKAGES AND DOCUMENT CONFIGURATIONS
%----------------------------------------------------------------------------------------

\documentclass{article}

\usepackage{graphicx} % Required for the inclusion of images

\usepackage[utf8]{inputenc}
\usepackage[OT4,plmath]{polski}

\usepackage{caption}
\usepackage{subcaption}
\usepackage{amsmath,amssymb,amsfonts,amsthm,mathtools}

\usepackage{hyperref}
\usepackage{url}
\usepackage{array}
\usepackage[flushleft]{threeparttable}

\usepackage{comment}

\usepackage{listings, lstautogobble}
\usepackage{fancyhdr}
\usepackage{graphicx}
\usepackage{hyperref}

\newtheorem{theorem}{Twierdzenie}
\newcommand{\plot}[1] {
	\includegraphics[width=\textwidth]{#1}
}

\graphicspath{ {./} }
\setlength\parskip{3pt}
% \setlength{\tabcolsep}{8pt}
% \renewcommand{\arraystretch}{1.5}

%\usepackage{times} % Uncomment to use the Times New Roman font

%----------------------------------------------------------------------------------------
%	DOCUMENT INFORMATION
%----------------------------------------------------------------------------------------

\title{\LARGE\textbf{Pracownia z analizy numerycznej} \\ Sprawozdanie do zadania \textbf{P2.2} \\
\vskip 0.2cm \large Prowadzący: dr Rafał Nowak\\
\author{ Jakub \textsc{Zadrożny}, Mateusz \textsc{Hazy}}}
\date{Wrocław, \today}

\begin{document}
\maketitle

\section{Wstęp}
Precyzyjne wyznaczanie położenia na Ziemi jest problemem, z którym ludzkość zmaga się od setek lat.
Na początku wykorzystywano zupełnie prymitywne metody (jak obserwacja otoczenia), z czasem wypracowano bardziej
zaawansowane sposoby (np. oparte na obserwacjach astronomicznych), aż w końcu opracowano rozwiązanie niemal doskonałe.
Rozwiązaniem tym jest system GPS (\textit{Global Positioning System}), czyli program nawigacji satelitarnej,
pozwalający określać położenie obiektu z dokładnością do kilku metrów (czasem nawet dokładniej).
Składa się on z trzech segmentów: kosmicznego, naziemnego oraz użytkownika. Niniejsze sprawozdanie skupia się na problemach
segmentu użużytkownika i ma na celu przedstawienie oraz porównanie metod wyznaczania położenia obiektu na podstawie
informacji z segmentu kosmicznego.

\subsection{Opis problemu}
Na potrzeby doświadczeń przyjmiemy model, w którym informacja, jaką użytkownik otrzymuje od satelity,
składa się z 4 liczb: \\
\begin{itemize}
    \item $(X, Y, Z)$ - współrzędne satelity w momencie wysłania sygnału\\
	\item $T_w$ - czas wysłania sygnału \\
\end{itemize}
Niech $T_{o}$ będzie czasem odebrania sygnału przez użytkownika. Wtedy czas podróży sygnały wynosi $T_o - T_w$ i
przyjmując, że sygnał przemieszcza się stale z prędkością światła, możemy określić, w jakiej odległości od satelity
znajduje się odbiornik (w linii prostej). Innymi słowy, odbiornik leży na \textit{sferze} o środku w punkcie $(X, Y, Z)$
i promieniu $c(T_w-T_o)$. Jeżeli pozycję odbiornika oznaczymy przez $(x, y, z)$, to równanie tej sfery wygląda następująco
\[
(x - X)^2 + (y - Y)^2 + (z - Z)^2 = \big[c(T_{o} - T_{w})\big]^2
\]

Posiadając informacje z trzech satelit, otrzymalibyśmy teoretycznie równania trzech sfer, które przecinałyby się w
mniej więcej dwóch punktach, z których jeden leżałby w pobliżu Ziemi. Jednak w praktyce zegary satelit i odbiornika
nie są zsynchronizowane, przez co promień każdej sfery zostanie zostanie przekłamany i pozycja zostanie wyznaczona błędnie
(lub sfery w ogóle nie będą mieć punktów wspólnych). Rozwiązaniem tego problemu jest wprowadzenie dodatkowej zmiennej $t$
oznaczającej błąd zegara odbiorcy. Aby otrzymać jednoznaczne rozwiązanie potrzebujemy teraz dodatkowego równania.
Prowadzi to do układu
\begin{equation} \label{eq:main_system}
\begin{aligned}
    (x-x_1)^2 + (y-y_1)^2 + (z-z_1)^2 - \big[c(t_1-t)\big]^2 &= 0 \\
    (x-x_2)^2 + (y-y_2)^2 + (z-z_2)^2 - \big[c(t_2-t)\big]^2 &= 0 \\
    (x-x_3)^2 + (y-y_3)^2 + (z-z_3)^2 - \big[c(t_3-t)\big]^2 &= 0 \\
    (x-x_4)^2 + (y-y_4)^2 + (z-z_4)^2 - \big[c(t_4-t)\big]^2 &= 0
\end{aligned}
\end{equation}
gdzie $(x_i, y_i, z_i)$ to współrzędnego i-tego satelity w czasie wysłania sygnału, $t_i$ to czas podróży sygnału od
i-tego satelity oraz $c$ -- prędkość światła.
Celem niniejszego sprawodzdania jest rozważenie możliwości rozwiązania tego układu.

\section{Metody rozwiązywania układu}
Niech $d_i \coloneqq ct_i$, a pod zmienną $t$ podstawmy $ct$.
Wprowadźmy oznaczenia
\begin{align}
f_i(x, y, z, t) &\coloneqq (x-X_i)^2 + (y-Y_i)^2 + (z-Z_i)^2 - (D_i-t)^2 \\
\label{eq:F} F(x, y, z, t) &\coloneqq (f_1(x, y, z, t) \ f_2(x, y, z, t) \ f_3(x, y, z, t) \ f_4(x, y, z, t))^T
\end{align}
dla $1 \leq i \leq 4$, zgodnie z notacją układu (\ref{eq:main_system}), gdzie $T$ oznacza transpozycję.

\subsection{Metoda Newtona}
Opracujemy metodę rozwiązania układu na podstawie metody Newtona. Chcemy znaleźć miejsce zerowe funkcji
$F$ określonej przez (\ref{eq:F}). Załóżmy, że mamy n-te przybliżenie wyniku $x_n \in \mathbb{R}^4$.
Checmy znaleźć takie $h \in \mathbb{R}^4$, że dla $x_{n+1} = x_n + h$ zajdzie $F(x_{n+1}) = 0$.

\noindent Korzystając z uogólnionego twierdzenia Taylora otrzymujemy
\[
f_i(x+h) \approx f_i(x) + \nabla f_i(x) \circ h
\]
gdzie $x, h \in \mathbb{R}^4$.
Zatem
\[
F(x+h) \approx F(x) + \mathbf{J}(x)h
\]
gdzie $\mathbf{J}$ to jakobian funkcji $F$.

\noindent Rożniczkując funkcje $f_i$ po kolejnych zmiennych otrzymujemy, że dla $p \in \mathbb{R}^4,\ p=(x \ y \ z \ t)^T$
\[
\mathbf{J}(p) = 2
\begin{pmatrix}
    x-X_1  & y-Y_1 & z-Z_1 & T_1-t \\
    x-X_2  & y-Y_2 & z-Z_2 & T_2-t \\
    x-X_3  & y-Y_3 & z-Z_3 & T_3-t \\
    x-X_4  & y-Y_4 & z-Z_4 & T_4-t
\end{pmatrix}
\]

\noindent Aby sprawdzić, kiedy $F(x_{n+1}) = 0$, rozwiążemy układ równań liniowych
\begin{align*}
    0 &= F(x_n) + \mathbf{J}(x_n)h \\
    -F(x_n) &= \mathbf{J}(x_n)h \\
    h &= -\mathbf{J}(x_n)^{-1}F(x_n)
\end{align*}
Możemy określić $x_{n+1} \coloneqq x_n + h$.


\subsection{Metoda Algebraiczna}
Zastosujemy inne podejście (nieiteracyjne). Układ równań (\ref{eq:main_system}) w innej postaci
wygląda następująco

\begin{equation}
\begin{aligned}
    x^2 + y^2 + z^2 -2xX_1 -2yY_1 -2zZ_1 + c_1 = t^2 -2tD_1 \\
    x^2 + y^2 + z^2 -2xX_2 -2yY_2 -2zZ_2 + c_2 = t^2 -2tD_2 \\
    x^2 + y^2 + z^2 -2xX_3 -2yY_3 -2zZ_3 + c_3 = t^2 -2tD_3 \\
    x^2 + y^2 + z^2 -2xX_4 -2yY_4 -2zZ_4 + c_4 = t^2 -2tD_4
\end{aligned}
\end{equation}
gdzie $c_i=X_i^2+Y_i^2+Z_i^2-D_i^2$ dla $1 \leq i \leq 4$.

\noindent Po odjęciu czwartego równania stronami od pierwszych trzech otrzymamy
\begin{equation}
\begin{aligned}
    -2( x(X_1-X_4) + y(Y_1-Y_4) + z(Z_1-Z_4) ) + c_1 - c_4 = -2t(D_1-D_4) \\
    -2( x(X_2-X_4) + y(Y_2-Y_4) + z(Z_2-Z_4) ) + c_2 - c_4 = -2t(D_2-D_4) \\
    -2( x(X_3-X_4) + y(Y_3-Y_4) + z(Z_3-Z_4) ) + c_3 - c_4 = -2t(D_3-D_4)
\end{aligned}
\end{equation}

\noindent Jest to układ 3 równań liniowych na 4 zmiennych. Układ taki nie posiada jednozanczego rozwiązania,
ale -- o ile nie jest sprzeczny -- posiada rozwiązania parametryczne. W naszym przypadku (dla realnych danych)
powinno to być rozwiązanie zależne od jednego parametru. Oznacza to, że pozostałe zmienne można wyrazić
jako kombinacje liniowe tego parametru tak, aby dla dowolnej wartości parametru układ równań był spełniony.
Ponieważ dla faktycznych danych żadna zmienna nie powinna być z góry ustalona (na trzech równaniach),
to możemy założyć, że parametrem jest $t$. Wtedy dla pewnych rzeczywistych $a_x, a_y, a_z, b_x, b_y, b_z$ mamy
$$x = a_xt + b_x $$
$$y = a_yt + b_y $$
$$z = a_zt + b_z $$

\noindent Wówczas dla dowolnego $t$ zachodzi
$$-2( (a_xt+b_x)(X_1-X_4) + (a_yt+b_y)(Y_1-Y_4) + (a_zt+b_z)(Z_1-Z_4) + c_1 - c_4 =-2t(D_1-D_4) $$
$$-2( (a_xt+b_x)(X_2-X_4) + (a_yt+b_y)(Y_2-Y_4) + (a_zt+b_z)(Z_2-Z_4) + c_2 - c_4 =-2t(D_2-D_4) $$
$$-2( (a_xt+b_x)(X_3-X_4) + (a_yt+b_y)(Y_3-Y_4) + (a_zt+b_z)(Z_3-Z_4) + c_3 - c_4 =-2t(D_3-D_4) $$

\noindent Otrzymaliśmy zatem równości trzech funkcji liniowych dla każdego argumentu.

\noindent Stąd
\begin{equation}
\begin{aligned}
    a_x(x_1-x_4) + a_y(y_1-y_4) + a_z(z_1-z_4) &= D_1-D_4 \\
    a_x(x_2-x_4) + a_y(y_2-y_4) + a_z(z_2-z_4) &= D_2-D_4 \\
    a_x(x_3-x_4) + a_y(y_3-y_4) + a_z(z_3-z_4) &= D_3-D_4 \\
    -2(b_x(X_1-X_4)+b_y(Y_1-Y_4)+b_z(Z_1-Z_4)) &= c_4-c_1 \\
    -2(b_x(X_2-X_4)+b_y(Y_2-Y_4)+b_z(Z_2-Z_4)) &= c_4-c_2 \\
    -2(b_x(X_3-X_4)+b_y(Y_3-Y_4)+b_z(Z_3-Z_4)) &= c_4-c_3
\end{aligned}
\end{equation}

\noindent Niech
\[
A \coloneqq
\begin{pmatrix}
X_1-X_4 & Y_1-Y_4 & Z_1-Z_4 \\
X_2-X_4 & Y_2-Y_4 & Z_2-Z_4 \\
X_3-X_4 & Y_3-Y_4 & Z_3-Z_4
\end{pmatrix}
\]

\noindent Wtedy
\begin{equation}
\begin{aligned}
    A
    \begin{pmatrix}
        a_x \\ a_y \\ a_z
    \end{pmatrix}
    &=
    \begin{pmatrix}
        D_1-D_4 \\ D_2-D_4 \\ D_3-D_4
    \end{pmatrix}
    \\
    A
    \begin{pmatrix}
        b_x \\ b_y \\ b_z
    \end{pmatrix}
    &=
    -\frac{1}{2}
    \begin{pmatrix}
        c_4-c_1 \\
        c_4-c_2 \\
        c_4-c_3
    \end{pmatrix}
\end{aligned}
\end{equation}

\noindent Z ostatnich dwóch równości możemy łatwo wyznaczyć współczynniki $a_x, a_y, a_z, b_x, b_y, b_z$.
Aby otrzymać konkretne rozwiązanie, możemy podstawić otrzymane zależności do czwartego równania.
Wtedy otrzymamy równanie kwadratowe jednej zmiennej następującej postaci
\begin{equation} \label{eq:quadratic}
\begin{split}
  t^2(a_x^2&+a_y^2+a_x^2-1) + \\
  &+ 2t(a_x(b_x-X_4)+a_y(b_y-Y_4)+a_z(b_z-Z_4)-D_4^2) + \\
  &+ (b_x-X_4)^2+(b_y-Y_4)^2+(b_z-Z_4)-D_4^2 = 0
\end{split}
\end{equation}

\noindent Po rozwiązaniu równania (\ref{eq:quadratic}) otrzymamy dwie możliwości na $t$, które wyznaczą
dwa rozwiązania $x, y, z, t$. W rozwiązaniu przyjęto, że szukanym wektorem jest ten o
mniejszym bezwzględnym błędzie zegara.

\subsection{Problemy}

\section{Uogólnione metody}
W praktyce użytkownik ma dostęp do więcej niż 4 satelit. Pobierając informacje z $n > 4$ satelit otrzymujemy układ równań
$$ f_{i}(x, y, z, t) = 0 \ dla \ i = 1,2...n $$

Powyższy układ może być sprzeczny, dlatego celem jest znalezienie $ (x, y, z, t) $ takich, że

$$\sum_{1}^{n} f_{i}^2(x, y, z, t) $$

będzie minimalna.

\subsection{Metoda najmniejszych kwadratów}
Zastosujemy metodę będącą uogólnieniem metody Newtona opisanej powyżej. \newline

Niech
$$ h_{n} = -(J_{n}^T J_{n})^{-1} J^T F(x_{n})$$

gdzie \newline

$J_{n}$ - macierz pochodnych cząstkowych w punkcie $x_n$ \newline

$J_{n}[i, j] = \frac{\partial f_i}{\partial x_j} $ \newline

$F(x) \in R^4 \to R^n , F(x) = \big[f_{i}(x)\big]$ \newline

Wtedy
$$ x_{n+1} = x_{n} + h_{n} $$
Jest n-tym przybliżeniem metody.

\subsection{Heurystyka}
Mamy układ równań
$$ f_{i}(x, y, z, t) =0 \ dla \ i = 1,2,3,4$$
Sprowadzamy do układu równań liniowych odejmując ostatnie równanie od pozostałych
$$-2x(x_i-x_4) -2y(y_i-y_4) -2z(z_i-z_4) + x_i^2 + y_i^2 + z_i^2 - (x_4^2 + y_4^2 + z_4^2) = c^2(-2t(t_i-t_4)+t_i^2-t_4^2)$$

Inaczej
$$
\begin{pmatrix}
x_1-x_4 & y_1-y_4 & z_1-z_4 & t_4-t_1 \\
x_2-x_4 & y_2-y_4 & z_2-z_4 & t_4-t_2 \\
x_3-x_4 & y_3-y_4 & z_3-z_4 & t_4-t_3
\end{pmatrix}
\begin{pmatrix}
x \\ y \\ z \\ t
\end{pmatrix}
=
-\frac{1}{2}
\begin{pmatrix}
-(x_1^2+y_1^2+z_1^2-t_1^2)+c_1 \\
-(x_2^2+y_2^2+z_2^2-t_2^2)+c_1 \\
-(x_3^2+y_3^2+z_3^2-t_3^2)+c_1 \\
\end{pmatrix}
$$
gdzie $c_1=x_4^2+y_4^2+z_4^2-t_4^2$.

\section{Testy}

Na poszczególne metody mają wpływ czynniki takie jak:
	\begin{itemize}
		\item wielkość błąd zegara
		\item liczba iteracji
		\item liczba satelit, z których została pobrana informacja
	\end{itemize}
W praktyce na dokładność pomiaru istotny wpływ mają czynniki zewnętrzne:
	\begin{itemize}
		\item opoźnienie światła w troposferze
		\item interferencja fal elektromagnetycznych
	\end{itemize}
Z tego powodu sprawdzanie dokładności metod zostało podzielone na badanie wpływu poszczególnych czynników na precyzję oraz uproszczoną symulację określania położenia w warunkach naturalnych.

\subsection{Symulacja ścieżki}

\par Do sprawdzenia dokładności metod została stworzona symulacja poruszania się użytkownika po losowej ścieżce na małym obszarze Ziemi. Model został uproszczony. Satelity były nieruchome, a błąd powodowany czynnikami zewnętrznymi symulowany był losowym opóźnieniem prędkości światła.


\subsection{Błąd zegara}



\subsection{Liczba iteracji}


\subsection{Liczba satelit}

Zaletą metody najmniejszych kwadratów jest rosnąca odporność na błedy wynikające z czynników zewnętrznych. Potencjalny duży błąd informacji z jednej z satelit jest niwelowany dokładnymi informacjami z pozostałych.

\section{Wnioski}


\end{document}
